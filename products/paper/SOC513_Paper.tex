%%%%%%%%%%%%%%%%%%%%%%%%%%%%%%%%%
% This is a slightly modified template of the one built by
% Steven V. Miller. Information can be found here:
%  http://svmiller.com/blog/2016/02/svm-r-markdown-manuscript/
%
% I added the use of raggedright to the anonymous option
% because journals,  the ability to put all the footnotes
% in endnotes, and the ability to manually adjust
% the starting page from the YAML header.
%
% Here are the options that you can define in the YAML
% header.
%
% fontfamily - self-explanatory
% fontsize - self-explanatory (e.g. 10pt, 11pt)
% anonymous - true/false. If true, names will be supressed and the
%                       text will be double-spaced and ragged
%                       right. For submission. 
% endnotes - true/false. If true, the footnotes will be put in a
%                   section at the end just ahead of the references.  
% keywords - self-explanatory
% thanks - shows up as a footnote to the title on page 1
% abstract - self explanatory
% appendix - if true, tables and figures will have  in
%                   front
% appendixletter - The letter to append to tables and figures in
%                             appendix
% pagenumber - Put in a number here to get a starting page number
%                         besides 1. 
%%%%%%%%%%%%%%%%%%%%%%%%%%%%%%%%%%


\documentclass[11pt,]{article}
\usepackage[left=1in,top=1in,right=1in,bottom=1in]{geometry}
\usepackage{amsmath}
\usepackage{float}
\usepackage{dcolumn}

\newcommand*{\authorfont}{\fontfamily{phv}\selectfont}
\usepackage[]{mathpazo}


  \usepackage[T1]{fontenc}
  \usepackage[utf8]{inputenc}



\usepackage{abstract}
\renewcommand{\abstractname}{}    % clear the title
\renewcommand{\absnamepos}{empty} % originally center

\providecommand{\tightlist}{%
  \setlength{\itemsep}{0pt}\setlength{\parskip}{0pt}}

\renewenvironment{abstract}
 {{%
    \setlength{\leftmargin}{0mm}
    \setlength{\rightmargin}{\leftmargin}%
  }%
  \relax}
 {\endlist}

\makeatletter
\def\@maketitle{%
  \newpage
%  \null
%  \vskip 2em%
%  \begin{center}%
  \let \footnote \thanks
    {\fontsize{18}{20}\selectfont\raggedright  \setlength{\parindent}{0pt} \@title \par}%
}
%\fi
\makeatother




\setcounter{secnumdepth}{0}



\title{Homeownership as a key for immigrants integration \thanks{Thanks to Aaron for all your kind help and patience.}  }



\author{\Large Pamanee Chaiwat\vspace{0.05in} \newline\normalsize\emph{University of Oregon, School of Architecture \& Environment}  }


\date{}

\usepackage{titlesec}

\titleformat*{\section}{\normalsize\bfseries}
\titleformat*{\subsection}{\normalsize\itshape}
\titleformat*{\subsubsection}{\normalsize\itshape}
\titleformat*{\paragraph}{\normalsize\itshape}
\titleformat*{\subparagraph}{\normalsize\itshape}


\usepackage{natbib}
\bibliographystyle{./resources/ajs.bst}


%\renewcommand{\refname}{References}
%\makeatletter
%\renewcommand\bibsection{
%    \section*{{\normalsize{\refname}}}%
%}%
%\makeatother

\newtheorem{hypothesis}{Hypothesis}
\usepackage{setspace}

\makeatletter
\@ifpackageloaded{hyperref}{}{%
\ifxetex
  \usepackage[setpagesize=false, % page size defined by xetex
              unicode=false, % unicode breaks when used with xetex
              xetex]{hyperref}
\else
  \usepackage[unicode=true]{hyperref}
\fi
}
\@ifpackageloaded{color}{
    \PassOptionsToPackage{usenames,dvipsnames}{color}
}{%
    \usepackage[usenames,dvipsnames]{color}
}
\makeatother
\hypersetup{breaklinks=true,
            bookmarks=true,
            pdfauthor={Pamanee Chaiwat (University of Oregon, School of Architecture \& Environment)},
             pdfkeywords = {immigrants, homeownership, assimilation, integration, Latinos, Hispanic},  
            pdftitle={Homeownership as a key for immigrants integration},
            colorlinks=true,
            citecolor=blue,
            urlcolor=blue,
            linkcolor=magenta,
            pdfborder={0 0 0}}
\urlstyle{same}  % don't use monospace font for urls

\usepackage{endnotes}


\newlength{\normalparindent}
\setlength{\normalparindent}{\parindent}

%prettier captions for figures and tables
%I am making the text of figure captions smaller but not table captions
\usepackage[labelfont=bf,labelsep=period]{caption}
\captionsetup[figure]{font=footnotesize}

\begin{document}
	
% \pagenumbering{arabic}% resets `page` counter to 1 
%
\setcounter{page}{1}

% \maketitle

{% \usefont{T1}{pnc}{m}{n}
\setlength{\parindent}{0pt}
\thispagestyle{plain}
{\fontsize{18}{20}\selectfont\raggedright 
\maketitle  % title \par  

}

{
   \vskip 13.5pt\relax \normalsize\fontsize{11}{12} 
\textbf{\authorfont Pamanee Chaiwat} \hskip 15pt \emph{\small University of Oregon, School of Architecture \& Environment}   

}

}







\begin{abstract}

    \hbox{\vrule height .2pt width 39.14pc}

    \vskip 8.5pt % \small 

\noindent Homeownership has been used as an indication of acquiring American
Dreams in many immigrants studies. This study examines the relationship
between immigrants experiences, identity and their homeownership and
nativity status. It reveals that even though immigrants are likely to
perceive themselves as ``typical American'' if they own a home, the
homeownership has insignificant impact to their experience in the US or
even worsen in some cases. However, owning a home in the US indicates a
weaker family ties in immigrants' family and thus signifying permanent
settlement in the US rather than their country of origin.


\vskip 8.5pt \noindent \emph{Keywords}: immigrants, homeownership, assimilation, integration, Latinos, Hispanic \par

    \hbox{\vrule height .2pt width 39.14pc}



\end{abstract}


\vskip 6.5pt

\noindent  \hypertarget{introduction}{%
\section{Introduction}\label{introduction}}

Migrations define many challenges for architects and planners.
Immigrants play important roles in urban changes specifically in labor
market shift, economic growth, settlement and residential patterns, as
well as cultural affluence. The controversy of racial segregation and
discrimination in the US creates opposition against immigrants which
leads to unachievable integration regardless of their contributions to
US economy. Within the scope of architecture scholarship, a counter
question must be asked: how have built environment and a city play role
in facilitating integration of these population. To start this
examination, the study begins will a statistical studies of immigrants'
homeownership in the US. Homeownership not only symbolizes the American
dream but also signifies economic competency which in many cases
achievable through education attainment, language proficiency, and
attachment to new host society.

The purpose of this study is to explore the relationship of
homeownership and how have it accelerated education and economic
integration of immigrant's families into American Society. Derive from
social sciences researches which look at how have immigrants of
different racial groups varied in homeownership or how have time of
entry the US affected the accessibility to buy a house, this study;
however, intents to point out the significant of home as a tool for
integration. It is intentionally overlook crowding and English
proficiency aspects as culture can become the key influencer of a family
structure or language used than socioeconomic constraints. Nevertheless,
the significant of homeownership has been a gap in previous studies and
is itself important in strategize the integration.

\hypertarget{background}{%
\section{Background}\label{background}}

Mayer and Lee's double cohort analysis of homeownership reports the
differences among native born and foreign born cohort. They evaluate the
effect of income, education, language proficiency, family status, and
their homeownership attainment (1998). Alba and Logan (1992) conducted a
homeownership research across twelve ethnic groups. They explain that
homeownership illustrate wealth accumulation associated with both
socioeconomic and residential mobility. Additionally, house is a
transferable asset benefitting their next generation. They also
differentiate homeownership as an indicator of assimilation or
stratification base on location and ethnic cohesion. Their findings
conclude that homeownership contributes to a process of incorporation of
minority groups. In alignment with Alba and Logan, ``Assimilation
today'' reports a better education attainment and occupational choice of
second generation Latinos. These rate increase similar to homeownership
rate that rise from 9.3\% in 1990 to 58\% in 2008 (Mayers \& Pitkin,
2010). They also suggest that integration happens at a faster rate when
neighborhood residents have access to citizenship and homeownership.

Using homeownership as an indicator of assimilation that accelerate
integration, this paper examines the statistical analysis of immigrants
experience between those who own a home and do not. Investigating how
have homeownership influences these factors and their trajectories will
helps identify housing strategy.

\hypertarget{data-and-methods}{%
\section{Data and Methods}\label{data-and-methods}}

Data in this study comes from Pew Hispanic Center 2011 National Survey
of Latinos (NSL2011).The study focuses at Latinos immigrants includes
their political views, values, attitudes towards immigration laws,
identity and assimilation. The sample size of this survey is 1,220
respondants with 492 native born and 728 foreign born.

Data from PEW has missing values. However, there is no impute value in
this study as proportional odds logistic regression are not nested and
look at each experience individually.

\begin{verbatim}
## 
## \begin{table}
## \caption{Home Ownership and Immigrants Experience}
## \begin{center}
## \begin{tabular}{l c c c c c }
## \hline
##  & Model 1 & Model 2 & Model 3 & Model 4 & Model 5 \\
## \hline
## home\_ownOwn         & $0.24973$       & $-0.11570$     & $0.24160$   & $0.27648$       & $-0.24042$     \\
##                      & $(0.14494)$     & $(0.12296)$    & $(0.20146)$ & $(0.14404)$     & $(0.12275)$    \\
## nativityForeign-born & $0.27315$       & $-0.31658^{*}$ & $-0.07254$  & $-0.45590^{**}$ & $-0.22644$     \\
##                      & $(0.14722)$     & $(0.12806)$    & $(0.20733)$ & $(0.15400)$     & $(0.12753)$    \\
## I(age - 40)          & $0.01825^{***}$ & $-0.00180$     & $0.00432$   & $-0.01310^{*}$  & $0.01451^{**}$ \\
##                      & $(0.00539)$     & $(0.00488)$    & $(0.00760)$ & $(0.00590)$     & $(0.00486)$    \\
## I((age - 40)^2)      & $-0.00046$      & $0.00039$      & $-0.00022$  & $0.00043$       & $0.00017$      \\
##                      & $(0.00023)$     & $(0.00021)$    & $(0.00030)$ & $(0.00024)$     & $(0.00021)$    \\
## \hline
## AIC                  & 1589.16211      & 2200.35789     & 907.87148   & 1651.42055      & 2231.45680     \\
## BIC                  & 1618.80925      & 2229.94672     & 937.60504   & 1681.10240      & 2261.08651     \\
## Log Likelihood       & -788.58105      & -1094.17894    & -447.93574  & -819.71027      & -1109.72840    \\
## Deviance             & 1577.16211      & 2188.35789     & 895.87148   & 1639.42055      & 2219.45680     \\
## Num. obs.            & 1034            & 1024           & 1049        & 1040            & 1031           \\
## \hline
## \multicolumn{6}{l}{\scriptsize{$^{***}p<0.001$, $^{**}p<0.01$, $^*p<0.05$}}
## \end{tabular}
## \label{table:coefficients}
## \end{center}
## \end{table}
\end{verbatim}

\hypertarget{model-1-immigrants-experience-in-the-treatment-of-the-poor.}{%
\subsection{Model 1: Immigrants Experience in the Treatment of the
Poor.}\label{model-1-immigrants-experience-in-the-treatment-of-the-poor.}}

Model 1 looks specifically at immigrants' perspective on the treatment
of the poor comparing their experience in their home country and in the
US. Homeownership or nativity do not have any significance in their
experience.

\begin{figure}

{\centering \includegraphics{SOC513_Paper_files/figure-latex/figure 1-1} 

}

\end{figure}

\hypertarget{model-2-immigrants-experience-in-the-moral-values-of-society}{%
\subsection{Model 2: Immigrants Experience in the Moral Values of
Society}\label{model-2-immigrants-experience-in-the-moral-values-of-society}}

Model 2 looks at immigrants' perspective on the moral values of society
comparing their experience in their home country and in the US. Nativity
impacts their moral values experience. In contrast, homeownership does
not have positive impact in their perception. Immigrants rather perceive
that the society is not better than their home country.

\begin{figure}

{\centering \includegraphics{SOC513_Paper_files/figure-latex/figure 2-1} 

}

\end{figure}

\hypertarget{model-3-immigrants-experience-in-the-moral-values-of-society}{%
\subsection{Model 3: Immigrants Experience in the Moral Values of
Society}\label{model-3-immigrants-experience-in-the-moral-values-of-society}}

Model 3 looks specifically at immigrants' perspective on opportunity to
get ahead comparing their experience in their home country and in the
US. Neither homeownership nor nativity impact their opportunity
experience. Regardless of the homeownership and nativity, the
respondants strongly experience better experience in the US.

\begin{figure}

{\centering \includegraphics{SOC513_Paper_files/figure-latex/figure 3-1} 

}

\end{figure}

\hypertarget{model-4-immigrants-experience-in-the-condition-of-raising-children}{%
\subsection{Model 4: Immigrants Experience in the condition of raising
children}\label{model-4-immigrants-experience-in-the-condition-of-raising-children}}

Model 4 looks specifically at immigrants' perspective on the condition
of raising children comparing their experience in their home country and
in the US. Nativity greatly effects their experience in raising
children. This illustrates the disadvantages immigrants face raising
their children if they are foreign-born. Homeownership does not have
impact or give any benefit to immigrants in raising their children.

\begin{figure}

{\centering \includegraphics{SOC513_Paper_files/figure-latex/figure 4-1} 

}

\end{figure}

\hypertarget{model-5-immigrants-experience-in-the-strengh-of-family-ties}{%
\subsection{Model 5: Immigrants Experience in the Strengh of Family
Ties}\label{model-5-immigrants-experience-in-the-strengh-of-family-ties}}

Model 5 compares their experience in family ties in their home country
and in the US. Both homeownership and nativity signify differences in
family ties experience. Owning home strongly indicates weaker family
ties even more than nativity. In this case, home ownership plays
important roles in illustrates the disadvantages immigrants face raising
their children if they are foreign-born. Homeownership does not have
impact or give any benefit to immigrants in raising their children.

\begin{figure}

{\centering \includegraphics{SOC513_Paper_files/figure-latex/figure 5-1} 

}

\end{figure}

\hypertarget{results-and-interpretation}{%
\section{Results and Interpretation}\label{results-and-interpretation}}

All five models in this study do not show any statistically significant
improvement in immigrants' experience in the case that they own a home.
Regardless of their homeownership and nativlity, the study shows that
the U.S. is seen as better than Latinos' countries of origin in many
ways. This is largely due to better opportunity in economy and raising
next generation. There are multiple possible explanations of this
outcome.

As the data for this study comes from the 2011 PEW National Hispanic
Survey, 2008 Economic regression possibly affected many immigrants'
experience regarding their experience in treatment of the poor (model
1), experience in opportunity to get ahead (model 3), and experience in
condition of raising children (model 4).

Interestingly, model 2: Immigrants Experience in the Moral Values of
Society illustrates that large number of respondents experience better
moral values in their home country than in the US if they own home. It
is possible that even if immigrants own a home, they experience spatial
seggregation and difficulty in morgage and loans throughout the process
of owning home. Additionally, owning a home can become financial burden
especially during economic crisis. However, regardless of their
experience in moral values owning home has critically improve the
possiblity of self perception as ``Typical American'' see figure 6.

Model 5: the experience in family ties has noticable difference if
immigrants own a home. As owning home has a symbolic meaning and
financial commitment of permanent settlement in one country, it is
reasonable to believe that homeownership effects the family ties in
their country of origin.

\begin{figure}

{\centering \includegraphics{SOC513_Paper_files/figure-latex/figure 6-1} 

}

\end{figure}

\begin{figure}

{\centering \includegraphics{SOC513_Paper_files/figure-latex/figure 7-1} 

}

\end{figure}

\hypertarget{conclusions}{%
\section{Conclusions}\label{conclusions}}

Homeownership remains the key signifier of self perception as acquiring
American Dreams. It symbolizes a compatibilities of foreign-born
immigrants to become a typical American. However, the ownerships do not
positively effect immigrants' experiences in the US. Longtitudinal study
would help examine the difference experiences prior and after owning
home. It would also portray comprehensive circumstance of immigrants
experience with less effects of particular economic downturn, in this
case the 2008 regression. Alternatively, crossectional study of
experience in the us among different ethnic groups can give a baseline
for comparison better than looking at one group alone.

Additionally, it is impossible to explicitly conclude the assimilation
without specific focus area such as economic assimilation or
acculturation. Homeownership plays important role as a highest goal of
American dream. However, further investigation of spatial integration or
seggregation as well as experience in discrimination must be
investigated. The decreasing property of immigrants experience in moral
values raise a question of immigrants' pathway in acquiring home, their
financial burden, household characteristic, and spatial integration.

Lastly, multiculturalism is the more prominant assimilation model in the
US than the melting pot model. This makes it possible for immigrants to
identify themselves as typical American while having distinct culture
than a typical American.

\hypertarget{references}{%
\section{References}\label{references}}

Alba, R., \& Logan, J. (1992). Assimilation and Stratification in the
Homeownership Patterns of Racial and Ethnic Groups. The International
Migration Review, 26(4), 1314-1341. \url{doi:10.2307/2546885}

Kochhar, R., Gonzalez-Barrera, A., \& Dockterman, D. (2009). Minorities,
Immigrants and Homeownership Through Boom and Bust. Pew Research Center.
Retrieve from
\url{https://www.pewhispanic.org/2009/05/12/through-boom-and-bust/}

Myers, D., \& Lee, S. (1998). Immigrant Trajectories into Homeownership:
A Temporal Analysis of Residential Assimilation. The International
Migration Review, 32(3), 593-625. \url{doi:10.2307/2547765}

Mayers, D., \& Pitkin, J. (2010). Assimilation Today: New Evidence Shows
the Latest Immigrants to America Are Following in Our History's
Footsteps. Center for American Progress. retrieve from
\url{https://dornsifecms.usc.edu/assets/sites/731/docs/Assimilation_Today-semifinal083010.PDF}

Papademetriou, G. D., Ray, B., \& Jachimowicz, M. (2004). Immigrants and
Homeownership in Urban America: An Examination of Nativity,
Socio-Economic Status and Place. Migration Policy Institute. retrieve
from
\url{https://www.migrationpolicy.org/research/immigrants-and-homeownership-urban-america-examination-nativity-socio-economic-status-and}



\bibliography{../project.bib}
\end{document}
